\clearpage
\hh*{其二}

\epi{彷徨者啊,那只现身于传说的乐土不可奢求啊!}

“在风纱废墟的西方有一片分裂大陆东西的山脉,山脉之上曾盘踞着远古的凶兽——噬虫,那是何等强大的力量,人类几乎没有与之抗衡的能力。但后来出现了一支受到风暴庇护的人类先祖,他们成功讨伐了噬虫解放了穿越山脉的途径,并给山脉命名为风暴之栖。在那高耸入云的山脉之巅一时间矗立起辉煌的堡垒,整个文明将唤起风暴的力量作为屏障在山巅抵御从大陆西边而来的外敌。”

“所以风纱圣城是风暴之栖被摧毁后流难的人为了缅怀失去的家园而修建的?”

“不对!风纱城是在湖中圣女之城,是风暴之栖在外敌式微后逐渐雄霸一方后,从大陆东方迁移过来的东古洛斯忒残族修建的。野心渐起的他们觊觎山脉的力量却难以抵御来自山脉的入侵,于是深入雾林寻求力量,他们的诚心不知为何感化了秘境湖之灵,于是湖之灵化作圣女的模样教给了他们抵御外敌的力量,也就是驾驭狂风的力量,当然这股力量已经随着风纱城一同消散了。风纱城建立后,由于心怀歹念的人被圣城使用狂风拒绝在外,人们便称这为风之面纱。”

“可既然风纱城拥有这灵赐的庇佑,最终又是如何毁灭的?”

“消亡当然是那曾经不可一世的傲慢带来的报应。”

一个年轻的声音插入话头来,众人的目光都开始寻找着发出这般异样言论的人的身影。

“一个从未目睹过圣城真容的无名之辈也敢这样诋毁先民吗?”

“在座的诸位都不曾见过自己口中高高在上的风纱城吧?”

“你又怎敢妄下断言?”

“诸位自诩先民后裔,那臭名昭著的殉葬仪式诸位不会不知道吧?”

……

她站在残垣断壁之间,原本高高耸立的钟楼现在只剩下一地灰烬,原本借风而起的屏障现在已平息于寂寥。她内心之中不由得升起一阵同情,但很快又被仇恨压制,那一幕幕惨状又重新出现在她眼前,她的脸上浮现出一丝扭曲的笑容,仿佛在嘲讽眼前的一切。

……

“绝对不能让任何人离开这座城!”

作为现任的风纱城城主,他从始至终都迷恋着那象征着奇迹与富饶的风纱城,他幻想着能有一日重现风纱城往日的辉煌。如今大敌当前,他笃信着那坚不可摧的风之领域,幻想着只要坚守城中,那么风纱城就永远不会覆灭。他下令禁止任何人离开风纱城,他一意孤行地想让所有人都追随他那即将破碎的意志。

他站在钟楼上,看着卫兵封锁着所有城门并且阻拦着每一个妄图逃离的人,他感到一阵心寒,同时有些恼怒,他实在不能理解那些逃难的人,他已经给了他们足够的希望和承诺了,剩下的只需要相信强大的魔法就行了,难道风纱城的城民不相信那由圣女所赐予的伟大力量吗?这强大的魔法保护这座城不止一次,每次都让风纱城绝处逢生,从前如此,将来也依然如此。

那城墙外卷起的强烈狂风,将沙土抛向空中形成一道厚重的屏障。这狂暴的风墙比任何人类守卫都要更加牢固,阻挡着外界的入侵但也阻碍了远眺的视线,从城中难以分辨出城外的景象。风中夹杂着的细小砂砾反射着太阳光,就像一道道刀痕遍布在风墙之上。城主等待着下属的传音,他将欣喜若狂地迎接着敌军被风墙撕裂的喜讯。

“城主。”

那熟悉的声音回到了他的耳边,他似乎难以掩盖声音中的些微颤抖。

“风墙依旧坚不可摧,是吗?”

可没等他收到回应,他忽然感到一阵穿刺的疼痛,一把匕首连根没入他的胸口,淋漓的鲜血随着剑身流动,又由刀尖滴下,溅起朵朵血花,顿时映出他心中泛起的巨大涟漪。他本以为他的信念如死水般寂静,可这一刻他才发现,他那抓狂的念想却如决堤的洪水,现在正从他胸口迸裂而出。

“为什么?”他浑身无法动弹,寒冷的刀尖渐渐因他的体温而变得炙热,让他的四肢都变得筋挛,这是他未曾体会过的麻木,带着他的思绪止不住地疯狂溃散。

“因为您会同这座城一起葬送在誓约众廷的征服之中。”身后之人掌控着他的身体,用着他最熟悉的声音冷静地说道。他也顾不上去看清这令他感到陌生的熟悉之人,他痛苦地思索着,仿佛在脑海里寻找着什么。

他脸上紧绷的肌肉变得松弛下来,他好像找到了。于是他问:

“风墙并非坚不可摧,是吗?”

身后之人没有回答,只用力地扭动了一下握在手上的匕首。这使得匕首彻底穿透了那脆弱的躯体,直至抵住別在那胸前的一块金属片。城主惨痛地嚎叫起来,他踉跄地跪倒在地上,转过头看清了那张他曾日日夜夜所见的面孔,他的贴身卫兵,他曾托付自己生命的人如今正掌控着他的生命。他身体抽动着想要伸出手抓向那匕首,但卫兵将匕首猛地向上一提,城主不由得挣扎着站立起来。

他默然,随后慢慢往前走去,卫兵猛然抽出匕首,他的身体扭曲地颤抖着,亦步亦趋地向前跌去。直到钟楼的墙边,他突然觉得一阵可笑,他向下看去,用颤巍巍的手抚摸着伤口,下面围拥着恼怒的城民,愤怒地向着钟楼嘶吼着。

“这风的面纱,终于要落下了吗?”

他向前坠落,在那一瞬间他看到了穿透风墙照射到城中的一缕阳光,还有那张他再熟悉不过的脸。那张脸带着一丝解脱的欢愉,看着前城主摔落在发狂的人群中被蜂拥而上的反叛者慢慢撕碎。卫兵敲响了钟声,随即便消失在钟楼之上。

意味着失守的钟声划破笼罩在城中的阴霾,仿佛闪电一样使得整座城沸腾起来。人群开始失魂落魄地四散逃离,你我裹挟中几乎碾碎了原本井然有序的房屋和石砖铺成的街道。沙石和瓦砾飞向空中,伴着一文不值的金银财宝被泼洒在奔走的人群中。所有人都在狂奔着,但似乎又说不上来是为了什么而奔跑,仿佛是压抑许久的精神在同一时刻爆发,近乎于失去理智的狂躁。

众疯魔似的人群从钟楼前退去,只留下遍地模糊的血肉,那象征风之庇佑的徽章在残忍的蹂躏下变得残破不堪,就像一块墓碑置于那块血肉之上。这时,另一群人围拥了过来,他们表情凝重,但既没有哀悼也没有悲痛,其中一人带着几乎对那血肉身份漠不关心的神情,庄重地弯下腰拾起了那褪去光泽的徽章。他转过身,高举徽章:

“余虽不受风之庇佑,而兹时受圣女之应允,以吾之血化汝之身形,愿汝唤风暴赐吾等安乐!”

那褪色的徽章上突然出现一道红色的光芒,在徽章的表面重新映照出那象征风之庇佑的图案。随着光芒进一步扩散,逐渐将众人与钟楼笼罩其中。

那些四散奔走的人们看着征兆奇迹的光芒正从钟楼前升起,于是渴望救赎的心愿又渐渐压制住了那破坏的欲望。城内所有人都不约而同地停下脚步跪在地上,等待着属于他们的救赎降临。

红光褪去,一阵强烈的风突然从钟楼前刮起,将围在钟楼前的众人都掀翻在地。待风逐渐平稳下来,那借沙土为媒介的巨大身形便出现在了众人的眼前。身形逐渐幻化出圣女上一次降临时的形态,如同一尊雕像盘踞在钟楼之上。肃穆与庄严下,她俯瞰着混乱不堪的圣城,接受着衣衫褴褛的逃亡者的祈祷。她发出刺耳的风声以此替她的意志传话,这风声刹那间传遍圣城的每一个角落,每一个人的耳中。这尖锐的吟唱刺痛着每个人的双耳,有人不堪其扰,鲜血从耳内溅出,在圣女的再临之歌中悲惨地死去。其他人目睹了这样的惨状后纷纷起身逃离,但他们的脚下都升起一股旋风将他们困在原地,使他们动弹不得。人们咒骂着,悲痛欲绝地挣扎着,却只能眼睁睁看着那圣女一点点地屠杀着满城百姓。

唯独钟楼前的众人不为所动,他们虔诚地跪在圣女的身躯下。这时,纷沓而至的马蹄声从城门外传来,众人仰头一看,那由风形成的屏障已经化为乌有。围攻圣城的军队正一马平川地向城内袭来,而在那之前,这座城的历史已然走到了尽头。手持徽章的人脸上终于露出了一丝笑容,他仰望着圣女和她身后久违的晴空,他缓缓闭上了眼。

“降下吧,风之面纱。”

……

她看着风暴中逐渐崩毁的圣城,神情恍惚地喘息着。刀刃一样的瓦片从她的脸颊旁划过,扑面而来的沙土笼罩在她无暇的肌肤上。她只能呆呆地望着,她不知道为什么,那无情的风似乎刻意避开了她。在整个圣城被撕裂的时候,她竟安然无恙地站立在瓦砾堆中,看着那些曾经金碧辉煌的砖瓦,那曾流光溢彩的珠宝玉石,那曾嵌在雕塑上的黄金,此刻都变得如废土一般,就像是那高高悬在半空的女神的玩物。

……

“那不也是传说吗?你还真把吟游诗人唱的玩意儿当真了啊?”

“是传说是真相,等你们哪天真有本事了自己去那片废墟瞧瞧就知道了!”

“年纪不大,口气不小啊。听你这么一说,你倒是去过了?你可见到了什么不得了的真相?”

“我,我,我……”

“小子,只会耍嘴皮子吹牛可不行,过往一些别有用心的人杜撰了不少先民们的历史,无一例外都是些添油加醋,颠倒黑白的胡言乱语。年轻人就别整天捣鼓那玄乎玩意,少搁那儿装什么深沉。再怎么说,死了的早就死了,所谓好与坏我们无从评判。说什么风纱后人溯源无根,漂泊将随着时间的终结而停止,我们这不还是各活各的,该干嘛干嘛。今天这个聚会,是应誓约众廷的要求商议新风纱城筹建的,而不是拿给你用来对什么人口诛笔伐的。我们已经是流浪者了,我们若是自己不愿背负起这责任,我们永远都将是你口中所谓的异乡人。”

……

大厅内的争吵在傍晚时分落下帷幕,对于所有与会的人来说,这无疑是一场失败的集会。在凄惨的夕阳下,圣女堂大厅再一次紧闭上了门扉,伫立在屋顶的圣女像仿佛溅出鲜血一般瘆人,如往日的许多时光内一样并没有祝福这群灰心丧气的失乡人。

“森万!”

人群都散去后,兀自倚在圣女堂门口的年轻人突然抬头看到一位同他年纪相仿的少女背着赤红的夕阳想他走来,她修长丰满的影子缓缓地伸向他的脚底。

“你又跟他们去吵了?”

“我只是告诉他们我所看到的而已。”

男孩伸出环抱着的双臂,脸却朝向另一边,仿佛在与其他人而不是正向他走来的少女说话。少女轻轻拽住他的手,温柔地看向少年的脸,说道:

“你不是说过要保守这个秘密的吗?”

“但我不忍心看着那群人一直被誓约众廷那些强盗蒙在鼓里,还借所谓的联合声明为由来满足他们自己的私欲。”

“但是你就算说了又能改变什么呢?更何况,要是让他们知道你通过自然之律假临了废墟,他们一定会把你交给律法肃正骑士团的。”

男孩抽出被女孩轻握的手,从颈上取下吊坠。他迎着夕阳将其托起,银灰色的琥珀在血色夕阳中闪烁,散发出黯淡的微光。琥珀正中心封存着一朵金色的芽孢,但穿过琥珀的光却让它显得毫无生气,更像是枯萎的花苞。

“伊薇,我自从得到这块吊坠,就已经想象过被骑士团剥皮抽筋的景象了。我想我要在那一刻到来之前,利用它让那些被虚伪的历史蒙蔽双眼的人清醒过来,告诉他们他们一直以来的信仰是错误的,错误的信仰只会白白抹除风纱城存在过的痕迹。”

女孩看着男孩的脸,尽管在夕阳下像沾满了鲜血,但这是她从未见过的刚毅与笃定,她在这一刻深深爱上了他面前这位不知死活的异乡人。就如她曾许诺的那样,她会永远伴随着虔诚者,成为其藤叶与荫庇。她深皱的额头舒缓了,眼神里带着一丝安抚,像是默认了他的理想,然后轻声说:

“我们回家吧。”

……

一座乳白色的高台矗立在视野中心,站在其上可以将整座城尽收眼底,星罗棋布的道路像一张巨网将大大小小的房屋收揽在其间,又仿佛水流向着这中心的高台聚集。高台之上是风之座,座下连接着百米高的阶梯,阶梯的两侧依次矗立着圆形石柱,石柱顶是盆装的盛风器皿,柱身上雕刻着风和圣女的形象:身披轻纱的圣女酣睡在风托起的枕上,静静地飘荡在空中。高台之下是由巨大的树根螺旋缠绕形成的支撑,每个交错的结点处彼此之间连接形成了螺旋上升的阶梯,一直通向高台正前方的巨大平台上。

这里是风纱城的中心,藉由圣洁使徒所创造出来的白色世界而隐藏在外人所无法触及的内里。繁荣的城市里汇聚着来自大陆各个角落的人们,有些是商人,有些是工匠,有些是祷告者,有些是魔法师,有些是仰慕者,有些是异端者,有些是渴望救赎的人,有些是追随爱情的人,有些是探寻真理的人。他们为这座城打造出了前无古人后无来者的奇观,他们让诗歌与音乐流淌在石砖路铺过的每一寸肌肤,他们用雕塑与绘画拼凑出高耸的玉宇与瑰丽的琼楼,他们让信仰的圣洁洗礼每一个人精神的污浊,他们用神秘莫测的洞察之术窥尽了造物之所在,他们用最世上最精妙的理论把律法镌刻在了不朽的石碑上,他们同这片恩赐的土地,一同构成了这座华美之城的伟大历史。

……

是风,还是火,亦或是借风肆虐的火。

哭声,呐喊声,亦或是掩埋于泪水中的不甘。

这是一幕再现了无数回的景象,在她脑海里就像来回旋转的木马,她只能袖手旁观着世界的崩毁,无数次地麻木于死亡的孤寂。不管是哪个世界,她背负着见证的诅咒,无法超脱这束缚,无法释怀那最初的罪孽,这是她自己选择的,一条将延伸至人类的消亡的道路。

……

森万搂抱着伊薇,他冰冷的身躯唯有这样才能勉强获得一丝温暖,他闭上眼,用脸颊机械地去触碰少女温和的手,他感到了一丝迟疑,那细腻的手似乎在慌乱地抚摸着他的脸,仿佛在质问着他。他被看穿了,他很生气,于是便松开了身边的女孩独自躺到了一边。但女孩却主动将他搂至怀中,像是母亲安抚孩子一般溺爱地刮蹭着他。伊薇将嘴唇贴在他的耳边,假装心不在焉地问:

“你今夜看到了什么?”

森万只是静静地感受着女孩嘴唇的湿润和冰凉,以及靠近他耳边的短促但有规律的呼吸。过了好一会儿他才缓过神来,回答道:

“我依附在一个少女的身体上,很难描述的感觉。”

“你感到难受吗?”

“嗯?”

“作为一个货真价实的男人,突然变成了女人,难道不会感到什么异样吗?”

“不要取笑我,伊。”

“我是认真的。毕竟现在你认知中是身为男性的一面战胜了身为女性的一面,但在你附身于那过去的少女时,你男性的一面一定会变得无比脆弱。”

“怎么会,那短暂的假临只不过是视觉上的连接,怎么会对我的认知产生影响。”

“你每次与过去的TA相连时,我都会默默待在你身旁观察你身体上的细微变化。我的直觉可是很敏锐的,你的表情和姿势总是会变得与平时不太一样,散发出来的气味也仿佛变了个人似的,或者说,在那时你就变成了你所依附的那个人。”

伊薇将双臂环抱在森万的腰间,仿佛在阻止眼前之人的消逝,她温柔得如同传说中的圣女,只此一刻把自己的所有倾注给了曾于她梦中现身的他,她认定的终将救赎她、救赎一切的使者。但她的呼吸却暴露了她的彷徨,她的身体与他相连的此刻,有种莫名的挣扎在让她抗拒着她的冲动,她无助地掌控着身体的主动,企图一点点地将森万的躯体蚕食。这不过是掩饰她脆弱内心的惶恐,将冷静的空气迅速升温,把逐渐升起的焦躁转变成了肉体的欢愉。

与木偶的缠绵是无趣的,它会任凭你如何支配它,哪怕你狠心地将它肢解它也不会发出一句怨言。因此大部分人会选择去那些可以花钱解决问题的地方,在他们享受的过程中他们通常倾向于认为这件事是双赢的,从而减轻来自内心的一点道德压力以及对这个社会的愧疚。总而言之,对于热恋中的情人来说,堵不如疏,这是伊薇在过往的日子里所理解到最深刻的一件事。

寒升露凝,霜月隐隐。森万已熟睡许久,伊薇离开他的身侧蜷缩在了床边。她透过小小的窗户望着宁静的街道,这是一处还算舒适的居所,并且破天荒地位于铎颂城的东南集市。或许这里曾是某个大商贾的宅邸,但随着岁月变迁,原主人或乔迁或死于战争,于是渐渐地这里被流落在集市附近的异乡人当做了庇护所。异乡人们在这里开办了公会,并且把宅邸的一楼大厅改造成了酒馆然后长久地经营了起来。与其说是酒馆,倒不如说是给那些瘾君子和酒鬼甚至一些精神堕落但颇有钱财的人提供了一个合法且安逸的场所来消遣。于是,同时充斥着严肃政论与歌舞喧嚣的宅邸变得闹热起来,在战争结束后的十几年里,这里成了集市上最耀眼的地方,也给公会灌入了不少新血液。

伊薇在常日里很少提到自己在这所宅邸曾经的过往,对森万的询问大多时候都以曾与某位肃正骑士有染搪塞过去,但她知道她到最后总会瞒不住真相,因此她内心始终背负着难以摆脱的负担,她知道此时的森万有更为重要的事要做,那么在他们逃离这里准备新生活之前,她不得不强忍着这份内疚的情绪。但她更加显露出来的却是尤为脆弱的,她自私的一面,她害怕自己失去森万。

她的身体如同皎洁的月光一样纯净,但她却几次三番地利用这身体来掩盖自我感情的流露,她愈是如此这样下去,她愈是痛苦地憎恨着自己,但那潜藏在人性深处的欲望却使她难以抗拒。

她从床沿站起身,小心翼翼地在窗台的书桌上点起一盏灯,然后从桌旁拿过一沓信纸,轻轻地坐在桌前,拿起一支墨迹斑斑的蘸水笔轻挑了两下只剩半瓶的蓝黑色墨水,再展平了一张算不上粗糙但还算崭新的羊皮纸,然后在信头写下了:

经由铎颂城,圣女堂公会,伊薇因藤萝启;

传至若曼笛城,卡西洛岛,伯特康托里收:

……

少女仿佛感到正有另一个人在透过她的双眼注视着这场如同儿戏一般终结的惨剧。她拼命地想要“看到”这不存在的另一个灵魂,她往废墟中央跑去,穿过那些视她无睹的呆若木鸡的士兵,她感到一定有所存在正潜伏在这片惨烈的土地上等待着她,或许是逃离这麻木漩涡的出口,或许是能告诉她一切答案的人,她神色慌乱,像一只受惊的鸟,一头扎进这于她来说“虚假”的征服军,在那无数双空洞的眼神中寻找着。

突然,她感受到了那真实存在的目光、那仿佛透过她的双眼注视世界的目光,此刻正站立在那崩塌的钟塔顶端,那目光也注意到了她,但他丝毫也没有诧异,而是选择空洞地无视她,她感到不解,于是她不管他正在向军队演讲着什么,而是自顾自地冲向了他,但还没等她靠近,她发现自己又回到了那片森林。前方是那片熟悉的湖泊,周围同样也弥漫着雾气,她绝望地向前跑,但这次她选择了向森林的边缘。