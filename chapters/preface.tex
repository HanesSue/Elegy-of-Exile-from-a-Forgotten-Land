\h*{写在最前}
\hh*{一些关于创作的琐碎}

对于正在写的,以及将要写的,最早的构想或许来自于高中时百无聊赖的生活。有时候,一个人总是要向外源源不断地产出些什么来排忧解闷,以至于不会让自己陷入内在的焦虑与痛苦。所以说,文字正是解决这些矛盾最好的药剂,因为它就像我们现在所说的脚本,万事万物所依靠的运行法则,足以撑起一个远比文字本身更为庞大的载体。我们现在看动漫、电视剧还有电影,这些都不是凭空而来光靠临时的言语造就的,而是利用了更立体的表现方式把扁平化的文字变得可视,变得容易接触了。阅读文字就像是一个解密的过程,一步一步地在脑海中重塑作者所传达的内涵,然后不断地再塑造与重构,最终变成了自己的一部分,于是便能切身地去体会字里行间原本空洞的密码,去构建自己的精神世界。

对于奇幻作品的喜爱,最初大概是来源于童话故事(严格来说其实其中大部分我都无法界定算不算我们如今所所谓的奇幻),无论是仙女教母还是侏儒怪,这些与现实世界格格不入的事物正是激发想象力的来源之一,或者说是最原始的憧憬,在孩童时期是对绮丽事物的喜爱,如今是对现实世界的失望。也正是随着自身慢慢成长,这份喜爱变得不再那么纯粹,或许大多数时候都是用来逃避现实的不快罢了。